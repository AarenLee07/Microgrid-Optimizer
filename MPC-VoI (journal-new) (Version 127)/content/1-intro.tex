
\subsection{Background}
Smart microgrids that incorporate distributed energy resources (DERs) play a crucial role in the pursuit of achieving carbon neutrality by integrating renewable energy generation \cite{li2023modeling}. However, this integration introduces significant challenges to the design, planning, and operation of the energy system. One prominent challenge is the growing temporal mismatch between renewable energy generation and peak electricity demand, known as the "Duck Curve" \cite{DuckC_ES}. The Duck Curve exhibits two distinct features. Firstly, there is a substantial decline in net demand (total electricity demand minus renewable generation) during midday when solar energy generation is high \cite{wang2023site}, represented by the curve's ``belly." This excess electricity production often leads to renewable curtailment. Secondly, there is a rapid increase in net demand, resembling the curve's ``head." Meeting this sudden demand requires grid operators to rely on large-scale fossil fuel generators as backup, which is costly and environmentally detrimental. This temporal imbalance strains the grid and poses challenges to the operation of smart grids \cite{widjaja2023general}.

Addressing the challenges posed by the duck curve involves implementing energy storage solutions such as electrical energy storage (e.g., batteries) or thermal energy storage (e.g., chilled water tanks or ice storage). However, integrating renewable generation and energy storage increases system complexity, necessitating dedicated algorithms for system operation. Advanced control algorithms like Model Predictive Control (MPC) \cite{kim2022site} and Reinforcement Learning (RL) \cite{touzani2021controlling} have been employed to optimize the coordination between building HVAC systems and energy storage, demonstrating superior performance compared to traditional rule-based control (RBC) strategies.

To optimize the design and operation of multiple heterogeneous but interconnected energy subsystems in an effective and reliable way is challenging \cite{Energy_model_complex}, as this optimization is information-intensive. The predicted information about the building load, renewable generation is needed in this predictive control paradigm. How to obtain and process the extensive information to make reasonable predictions in an online manner is crucial for making informed decisions in microgrids \cite{Intensive_data_grid}.

\subsection{Existing Works}

\subsubsection{Smart grid operation }

Smart grid operation involves the coordination of multiple energy systems, such as energy storage charging/discharging, electric vehicle charging, and space pre-cooling/pre-heating, in an efficient and reliable way \cite{dong2023modeling}. Within this domain, three main control paradigms are commonly employed: rule-based control (RBC), model-predictive control (MPC), and reinforcement learning (RL). RBC relies on a predefined set of rules to govern the operation of the smart grid. Examples of RBC strategies include maximizing self-consumption (MSC) and Time-of-Use (TOU) approaches discussed in \cite{zou2022comparative}. Currently, RBC strategies dominate the industry due to their simplicity and interpretability. However, a drawback of RBC is that the predetermined rules often fail to lead to optimal control actions.

MPC and RL are two emerging advanced control strategies in the field of smart grid management research and applications. MPC formulates an optimization control problem based on a physical model of the dynamic system and repeatedly re-optimizes as new system states are observed \cite{hu2021model}. In contrast, RL aims to learn a control strategy directly from reward maximization using a Markov decision process (MDP), without explicit knowledge of the system dynamics \cite{wang2020reinforcement}. Although both MPC and RL are optimization-based control strategies, they differ in their approach to decision-making. MPC relies on a model of the system dynamics to make informed decisions, while RL takes a model-free approach by relying on data. Compared to RL, MPC is considered more data-efficient and interpretable for energy system management \cite{wang2023comparison}. As a result, it has been extensively explored in the literature for various applications such as electric vehicle charging coordination \cite{shi2018model}, thermal storage operations \cite{tang2019model}, and hydrogen energy systems \cite{RenewMG_reduc_DC}. MPC has also been demonstrated in field studies, as exemplified by \cite{MPC_insitu}.

Within the microgrid context, economic Model Predictive Control (MPC) aims to optimize various control objectives, which may include utility costs, carbon emissions, system stability, and occupant satisfaction \cite{MEMS_LR}. Utility costs typically comprise two primary components: energy charges and demand charges. Energy charges are determined based on the amount of electricity consumed, and the pricing scheme may consider factors such as time of day, season, and overall consumption levels \cite{fernandez2023demand}. On the other hand, demand charges are calculated based on the maximum power demand observed by the consumer during a billing cycle, and for certain commercial electricity consumers, they can account for more than half of the total electric bill. Houben et al. \cite{uncertainty_dc} highlighted that the inability to accurately predict peak demand may lead to insufficient energy stored in the battery energy storage system (BESS) to avoid a new peak demand, resulting in additional costs.

\subsubsection{Building load prediction}

The initial step towards predictive control involves making predictions regarding building load and renewable generation. Building load prediction is an area of active research due to its broad applications in optimizing HVAC control \cite{kusiak2010cooling}, operating thermal energy storage systems \cite{luo2017data}, planning energy distribution systems\cite{pedersen2008load}, and managing smart grids \cite{xue2014interactive}. 

Approaches for forecasting building thermal load can be broadly classified into three categories: white-box physics-based models, gray-box reduced-order models, and black-box data-driven models\cite{wang2020building}. White-box models utilize detailed heat and mass transfer equations to predict building loads. Commercial software tools like EnergyPlus, Dest, and TRNSYS are available for setting up white-box models \cite{li2014review}. Black-box models, on the other hand, rely solely on historical data to predict building thermal load. Gray-box models simplify the building's thermal dynamics by using reduced-order Resistance and Capacity (RC) models, and then estimate parameter values using historical data. For smart grid operations, black-box data-driven models are widely preferred for two main reasons. Firstly, advancements in machine learning algorithms and the increased availability of historical data have made black-box models more accurate \cite{liu2023timetabling}. Secondly, developing physics-based models requires extensive input data, including thermal properties of the building envelope, occupancy and equipment schedules, which can be time-consuming, if not totally impossible, to gather. Some widely used black-box load prediction models will be introduced in greater details in Section. \ref{section:Building load forecasts}.

However, due to limited access to high-quality training data, potential changes in energy usage patterns, and system behaviors, uncertainties may arise in both demand and generation predictions \cite{tian2022daily}. Additionally, limited research directly addresses load prediction accuracy in specific downstream tasks. Studies have shown that under Model Predictive Control (MPC) or its surrogate neural networks (NN), the requirement for prediction accuracy appears to be moderate \cite{sun2016nonlinear, wu2022learning}. However, these evaluations do not consider demand charges. Gust et al. suggest that such conclusions may not hold when demand charges are taken into account \cite{gust2021strategies}. Stochastic programming is a general solution for addressing these challenges, but it often leads to a substantial increase in computational complexity.

\subsection{Scope and objectives}

Our literature review revealed a significant gap in the analysis of load forecast information's value in microgrid energy management, particularly in relation to the presence or absence of demand charges. This gap represents a notable disparity between two prominent clusters of research in the field. It is important to note that this concern has garnered attention from researchers within the broader context of \emph{integrated learning \& optimization} (ILO). For instance, Silwal et al. \cite{elmachtoub2022smart} propose a loss function, along with its convex surrogate, SPO+, which explicitly incorporates the optimization problem structure into the prediction loss. These pioneering works offer valuable intellectual insights; however, their applicability is unfortunately limited to a specific class of problems.

In this work, we aim at investigating the impact of building load forecasts on MPC of the microgrid, and examining the influences of demand charge in microgrid management. Our specific objectives are as follows:
\begin{enumerate}
    \item Perform a systematic review and residual analysis on cutting-edge and well-adopted load forecast models.
    \item Quantify the VoI of building energy management tasks through a realistic MPC pipeline under various scenarios. Specifically, we compare the cases with and without demand charge.
    \item Conduct parametric experiments to examine the correlations between the forecast residual patterns and downstream control performances. Specifically, we unveil the asymmetric effect of forecast errors.
    %\item Found that the impact of prediction errors are asymmetric: underestimating load leads to more sub-optimal control outcome than overestimating the load with the existence of demand charge.
    %\item Reveal that conventional lump metrics widely used in the forecasting tasks are not suitable for microgrid energy management under MPC, when demand charge is considered.
\end{enumerate}

The rest of the manuscript is organized as follows: Sec. \ref{sec:method} details the workflow of this study, including forecasting approaches, control model for microgrid energy management and configurations for the simulations. Then in Sec. \ref{sec:results}, we present the results of the simulations. In Sec. \ref{sec:discussion}, we discuss and extend the results, before we conclude in Sec. \ref{sec:conclusion}.
%Sec.\,\ref{sec:model} details the optimal control model for microgrid energy management, followed by Sec.\,\ref{sec:VoI} that discusses the value of building load information and the role of peak power tracking. We perform comprehensive numerical studies to demonstrate our findings and validate our proposed approach in Sec.\,\ref{sec:results}. The manuscript is concluded in Sec.\,\ref{sec:conclusion}.



