
In this research paper, we investigate the influence of forecast accuracy on the control performance in the context of smart grid operation. This is of utmost importance because the ultimate goal is to achieve effective control performance rather than merely accurate predictions. Furthermore, the presence of different pricing schemes adds complexity to optimizing control performance. Particularly, the pricing scheme that includes demand charges in the utility bill poses a significant obstacle. The longer billing cycle associated with demand charges compared to time-of-use costs makes it challenging to effectively control demand charges within a short optimization window for Model Predictive Control (MPC). Therefore, we thoroughly analyze the impact of prediction uncertainties with and without considering demand charges, and our findings reveal the following three key points:

\begin{enumerate}
    \item When demand charges are not taken into account, MPC achieves near-optimal performance even when the Mean Absolute Percentage Error (MAPE) of forecast reaches 10\%. Improving the accuracy of the forecasts has a limited contribution to enhancing control performance. Moreover, the simple Adaptive Moving Average (AMA) model proves to be an ideal forecasting model as it is computationally efficient, requires minimal historical data, and achieves satisfactory forecasting performance. However, when considering demand charges, even 1\% MAPE significantly deteriorates control performance to a level lower than Rule-Based Control (RBC).
    \item Through simulations involving artificial noise, including underestimation and overestimation, we uncover the asymmetric impact of forecasts when demand charges are considered. Underestimating future load leads to a substantial increase in peak demand, resulting in poor control performance.
    \item Singular lump metrics such as MAPE cannot serve as reliable indicators of MPC control performance, especially when demand charges are involved. These metrics overlook the residual distribution patterns and temporal correlations, which hold valuable information for MPC. Therefore, more intricate forecast features should be considered, aiming to provide a clear objective for upstream forecasting tasks.
\end{enumerate}

Based on these findings, it is important to acknowledge certain limitations in this study. The testbed formulated in this research is relatively simplistic, as it does not account for electric vehicles, despite the growing trend of incorporating charging stations in microgrid management. Additionally, we did not investigate the battery degradation in the Battery Energy Storage System (BESS). Furthermore, we only examined a few system configurations, which may limit the generalizability of our findings.

In addition to the MPC strategy implemented in this study, there are several solutions available to address prediction uncertainties, such as robust MPC and stochastic MPC. As a next step, we plan to evaluate their performance. Both robust MPC and stochastic MPC incorporate forecast uncertainty into the optimization stage, which may help mitigate the asymmetric impact of forecasting errors revealed in this study.
%Another concentration for the future work is to further explore a synthesized metric for evaluating the forecasts, 
